\documentclass{article}
\usepackage[a4paper,left=3.5cm,right=2.5cm,top=2.5cm,bottom=2.5cm]{geometry}
\usepackage[MeX]{polski}
\usepackage[cp1250]{inputenc}
%%\usepackage{polski}
%%\usepackage[utf8]{inputenc}
\usepackage[pdftex]{hyperref}
\usepackage{makeidx}
\usepackage[tableposition=top]{caption}
\usepackage{algorithmic}
\usepackage{graphicx}
\usepackage{enumerate}
\usepackage{multirow}
\usepackage{amsmath} %pakiet matematyczny
\usepackage{amssymb} %pakiet dodatkowych symboli
\begin{document}


Tu umieszczamy kod TeXa, ktory bedzie kompilowany, $a^2$ a suma $\sum_{i=0}^{\infty}{2^i}$

\begin{displaymath}
\sum_{1=0}^{\infty}{2^i}
\end{displaymath}
\begin{equation}
\label{eq:iloczyn}
\prod^{n=i^2}_{i=2}=\frac{\lim^{n\rightarrow4}(1+\frac{1}{n})^n}{\sum k (\frac{1}{n})}
\end{equation}
Na podstawie r�wnania \ref{eq:iloczyn}
\begin{equation}
\left[
\begin{array}{cccc}
a_{11} & a_{12} & \ldots & a_{1K} \\
a_{11} & a_{12} & \ldots & a_{2K} \\
\vdots & \vdots & \ddots & \vdots \\
a_{K1} & a_{K2} & \ldots & a_{KK} \\
\end{array}
\right]
*
\left[
\begin{array}{c}
x_1 \\
x_2 \\
\vdots \\
x_K \\
\end{array}
\right]
=
\left[
\begin{array}{c}
b_1 \\
b_2 \\
\vdots \\
b_K \\
\end{array}
\right]
\end{equation}
\begin{verbatim}
for(int i=0;i<40;i++)
 printf("Hello world");
\end{verbatim}
\begin{algorithmic}
\FOR{i=0,1,$\ldots$,40}
 \item{Wy�wietl napis "Hello World"}
\ENDFOR
\end{algorithmic}
\begin{displaymath}
[x]_{A}=\left\{{y}\in{U}:a(x)=a(y)},\forall{a}\in{A}\right\}, where~ the~ central~ object~ {x}\in{U}
\end{displaymath}
dodawa�~ albo komende \quad albo dodac to \ to \ to
\begin{displaymath}
P$\Big( A=2|\frac{[A]^2}{B}>4 )$\\
\end{displaymath}

\end{document}